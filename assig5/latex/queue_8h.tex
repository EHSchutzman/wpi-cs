\section{queue.h File Reference}
\label{queue_8h}\index{queue.h@{queue.h}}
\subsection*{Classes}
\begin{CompactItemize}
\item 
struct \bf{queue}
\end{CompactItemize}
\subsection*{Typedefs}
\begin{CompactItemize}
\item 
typedef \bf{queue} \bf{Queue}
\end{CompactItemize}
\subsection*{Functions}
\begin{CompactItemize}
\item 
\bf{Queue} $\ast$ \bf{create\_\-queue} (int max\_\-cells)
\item 
void \bf{delete\_\-queue} (\bf{Queue} $\ast$which\_\-queue)
\item 
int \bf{enqueue} (\bf{Queue} $\ast$which\_\-queue, void $\ast$ptr)
\item 
void $\ast$ \bf{dequeue} (\bf{Queue} $\ast$which\_\-queue)
\item 
void $\ast$ \bf{at\_\-queue\_\-top} (\bf{Queue} $\ast$which\_\-queue)
\end{CompactItemize}


\subsection{Typedef Documentation}
\index{queue.h@{queue.h}!Queue@{Queue}}
\index{Queue@{Queue}!queue.h@{queue.h}}
\subsubsection{\setlength{\rightskip}{0pt plus 5cm}typedef struct \bf{queue} \bf{Queue}}\label{queue_8h_6f23aa91ac8f9ef00f6b1c1a9d017e9a}




\subsection{Function Documentation}
\index{queue.h@{queue.h}!at_queue_top@{at\_\-queue\_\-top}}
\index{at_queue_top@{at\_\-queue\_\-top}!queue.h@{queue.h}}
\subsubsection{\setlength{\rightskip}{0pt plus 5cm}void$\ast$ at\_\-queue\_\-top (\bf{Queue} $\ast$ {\em which\_\-queue})}\label{queue_8h_70ca1dde791d8b919984ca5c98da8ade}


Peek at top of queue, without dequeing \begin{Desc}
\item[Parameters:]
\begin{description}
\item[{\em which\_\-queue}]Pointer to Queue you want to peek at. \end{description}
\end{Desc}
\begin{Desc}
\item[Returns:]Top entry of the queue, NULL if queue is empty. \end{Desc}
\index{queue.h@{queue.h}!create_queue@{create\_\-queue}}
\index{create_queue@{create\_\-queue}!queue.h@{queue.h}}
\subsubsection{\setlength{\rightskip}{0pt plus 5cm}\bf{Queue}$\ast$ create\_\-queue (int {\em max\_\-cells})}\label{queue_8h_b710f6637419a761df20688da7ece206}


Create a queue by allocating a Queue structure, initializing it, and allocating memory to hold the queue entries. \begin{Desc}
\item[Parameters:]
\begin{description}
\item[{\em max\_\-cells}]Maximum entries in the queue \end{description}
\end{Desc}
\begin{Desc}
\item[Returns:]Pointer to newly-allocated Queue structure, NULL if error. \end{Desc}
\index{queue.h@{queue.h}!delete_queue@{delete\_\-queue}}
\index{delete_queue@{delete\_\-queue}!queue.h@{queue.h}}
\subsubsection{\setlength{\rightskip}{0pt plus 5cm}void delete\_\-queue (\bf{Queue} $\ast$ {\em which\_\-queue})}\label{queue_8h_59e6e61741ed251ed46a6a38990c24b0}


Deletes the queue structure. \begin{Desc}
\item[Parameters:]
\begin{description}
\item[{\em which\_\-queue}]Pointer to Queue structure that has to be deleted. \end{description}
\end{Desc}
\index{queue.h@{queue.h}!dequeue@{dequeue}}
\index{dequeue@{dequeue}!queue.h@{queue.h}}
\subsubsection{\setlength{\rightskip}{0pt plus 5cm}void$\ast$ dequeue (\bf{Queue} $\ast$ {\em which\_\-queue})}\label{queue_8h_a150371d99ffab3f34b09efa4669a47b}


Removes element from beginning of queue, and returns it. \begin{Desc}
\item[Parameters:]
\begin{description}
\item[{\em which\_\-queue}]Pointer to Queue you want to remove element from. \end{description}
\end{Desc}
\begin{Desc}
\item[Returns:]First entry of the queue, NULL if queue is empty. \end{Desc}
\index{queue.h@{queue.h}!enqueue@{enqueue}}
\index{enqueue@{enqueue}!queue.h@{queue.h}}
\subsubsection{\setlength{\rightskip}{0pt plus 5cm}int enqueue (\bf{Queue} $\ast$ {\em which\_\-queue}, void $\ast$ {\em ptr})}\label{queue_8h_1d57a1b76a1967de5c006ed78777e65f}


Adds a pointer at the end of a Queue. \begin{Desc}
\item[Parameters:]
\begin{description}
\item[{\em which\_\-queue}]Pointer to queue you want to add element to. \item[{\em ptr}]Pointer to be pushed. \end{description}
\end{Desc}
\begin{Desc}
\item[Returns:]0 if successful, -1 if not. \end{Desc}
